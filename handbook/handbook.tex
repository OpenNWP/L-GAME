% This source file is part of the Geophysical Fluids Modeling Framework (L-GAME), which is released under the MIT license.
% Github repository: https://github.com/OpenNWP/L-GAME

\documentclass[10pt]{report}
\usepackage[utf8]{inputenc}
\usepackage{a4wide, amsmath, xcolor, longtable, geometry, fancyhdr, mathtools, array, listings}
\usepackage[style = numeric, backend = biber]{biblatex}
\usepackage{fouriernc}
\usepackage[T1]{fontenc}
\usepackage[hidelinks]{hyperref}
\geometry{a4paper, top = 15mm, left = 5mm, right = 5mm, bottom = 17mm}
\fancypagestyle{plain}{
\fancyhead[L]{\texttt{L-GAME} handbook}
\fancyhead[R]{\textsc{\texttt{L-GAME} development team}}
\fancyfoot[C]{\thepage}
\addtolength\footskip{12pt}}
\definecolor{table_green}{rgb}{0, 0.6, 0}
\title{\texttt{Limited-area Geophysical Fluids Modeling Framework (L-GAME)} Handbook}
\author{\texttt{L-GAME} Development Team}
\date{}
\newcommand{\md}[1]{\frac{D#1}{Dt}}
\newcommand{\omegabi}{\text{{\osgbi ω}}}
\newcommand{\mubi}{\text{{\osgbi μ}}}
\newcommand{\sigmabi}{\text{{\osgbi σ}}}
\newcommand{\epsilonbi}{\text{{\osgbi ϵ}}}
\newcommand{\etabi}{\text{{\osgbi η}}}
\newcommand{\zetabi}{\text{{\osgbi ζ}}}
\addbibresource{references.bib}
\DeclareFieldFormat[article]{title}{{#1}}

\begin{document}

\maketitle

\tableofcontents

\chapter{Overview}
\label{chap:overview}

\texttt{L-GAME} is the application of the theory and numerics of \texttt{GAME} to a quadrilateral latitude-longitude grid in a regional domain. The spatial and temporal discretizations are identical, apart from the following differences:
%
\begin{enumerate}
\item The domain is regional and not global (as mentioned).
\item The grid is a quadrilateral (rotated) latitude-longitude grid.
\item The connections of the gridpoints in longitude direction are small circles instead of great circles.
\end{enumerate}
%
This is only the handbook (manual) of \texttt{L-GAME}, it explains how to configure, compile and run (use) the model. For a scientific derivation of the model see \cite{kompendium} and the literature cited therein. The source code of the project is maintained on Github (\url{https://github.com/OpenNWP/L-GAME}).

\chapter{Code structure}
\label{chap:code_structure}

The code of the model resides in the directory \texttt{src}.

\section{Spatial operators}
\label{sec:spatial_operators}

\begin{itemize}
\item Coriolis: \cite{thuburn_f_discrete_plane} and \cite{ringler_trsk} modified by \cite{doi:10.1002/qj.3294}
\item kinetic energy: \cite{doi:10.1002/qj.1960}
\end{itemize}

\section{Time stepping}
\label{sec:time_stepping}

A fully Eulerian time stepping is employed. The basic building structure is a two-time-level predictor-corrector scheme. In the vertical, at every substep, an implicit column solver is used, which makes it possible to violate the CFL criterion of vertically propagating sound and fast gravity waves. This has the cost of decreasing the accuracy of these modes, which is however a bearable trade-off, since these waves are of low meteorological relevance. Furthermore, a forward-backward scheme is used, where the divergence term is backward.

\chapter{Installation}
\label{chap:installation}

It is recommended to run the model on Linux. These installation instructions are tested for Ubuntu, for other Linux distributions they might have to be modified.

\section{Dependencies}
\label{sec:dependencies}

The following dependencies must be installed before being able to successfully build the model:

\begin{itemize}
\item \texttt{sudo apt-get install gfortran make cmake wget python3-pip libnetcdff-dev}
\item Clone the RTE+RRTMGP repository: \texttt{git clone https://github.com/earth-system-radiation/rte-rrtmgp}
\end{itemize}

\section{Building}
\label{sec:building}

\texttt{CMake} is used for building \texttt{L-GAME}. Execute \texttt{./compile.sh} to build the model.

\chapter{Running the model}
\label{chap:running_the_model}

\section{Configuring the model domain}
\label{sec:configuring_the_model_domain}

\subsection{Vertical grid structure}
\label{sec:vertical_grid_structure}

The vertical grid structure is the same as in \texttt{GAME}, which is explained in \cite{game_handbook}.

\section{Dynamics configuration}
\label{sec:dynamics_configuration}

\section{Physics configuration}
\label{sec:physics_configuration}

\section{Coupling to the radiation field}
\label{sec:coupling_to_the_radiation_field}

\texttt{L-GAME} employs the so-called \texttt{RTE+RRTMGP (Radiative Transfer for Energetics + Rapid and Accurate Radiative Transfer Model for Geophysical Circulation Model Applications—Parallel)} \cite{doi:10.1029/2019MS001621}, \cite{rte-rrtmgp-github} scheme.

\section{Configuring Output}
\label{sec:configuring_output}

\chapter{Initialization}
\label{sec:initialization}

\chapter{Boundary conditions}
\label{chap:boundary_conditions}

\chapter{Physical surface properties}
\label{sec:physical_surface_properties}

The properties of the surface of the Earth influence the evolution of the atmospheric fields. Therefore, physical surface properties need to be obtained from external sources and interpolated to the model grid. The following fields are required:
%
\begin{itemize}
\item The land distribution. source: \url{https://ral.ucar.edu/sites/default/files/public/product-tool/noah-multiparameterization-land-surface-model-noah-mp-lsm/usgs/sfc-fields-usgs-veg30susgs.gz} \cite{glcc}
\item The orography. source: \url{https://www.ngdc.noaa.gov/mgg/global/relief/ETOPO1/data/ice_surface/grid_registered/netcdf/ETOPO1_Ice_g_gmt4.grd.gz} \cite{etopo1}, \cite{etopo1_add}
\item The lake fraction (share of a grid cell covered by lakes). source: \url{http://www.flake.igb-berlin.de/data/gldbv2.tar.gz} \cite{gldb}
\item The density of the soil.
\item The specific heat capacity of the soil.
\item The temperature diffusivity of the soil.
\item For NWP runs without coupling to an ocean model, the SST needs to be prescribed in order to calculate sensible and latent heating rates at the ocean surface (actually, we use the near sea surface temperature (NSST)). source: \url{https://nomads.ncep.noaa.gov/pub/data/nccf/com/nsst/prod/} (The SST is set analytically for idealized simulations.)
\item The land-sea mask of the NCEP NSST data. source: \url{https://downloads.psl.noaa.gov/Datasets/noaa.oisst.v2/lsmask.nc}
\end{itemize}

\appendix

\printbibliography

\end{document}













